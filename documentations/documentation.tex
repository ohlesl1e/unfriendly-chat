\documentclass[12pt]{article}
\usepackage[margin=1in]{geometry}
\usepackage{amsmath}
\usepackage{amssymb}
\usepackage{enumitem}
\usepackage{setspace}
\usepackage{hyperref}
\usepackage{indentfirst}
\usepackage[all]{nowidow}
%\newtheorem*{proposition}{Proposition}
\newcommand{\N}{\mathbb{N}}
\newcommand{\Z}{\mathbb{Z}}
\newcommand{\Q}{\mathbb{Q}}
\newcommand{\R}{\mathbb{R}}
\newcommand{\C}{\mathbb{C}}

\usepackage[
backend=biber,
style=numeric,
sorting=none
]{biblatex}

\addbibresource{mybib.bib}

\title{%
    Unfriendly Chat\\
    \large Team Wu-Tang LAN}
\author{Leslie Zhou (team lead)\\ Khanh Nguyen \\ Warren Singh}
\date{\today}

\begin{document}
\maketitle

\newpage
\tableofcontents
\newpage

\section{Introduction}
%Introduction/technical context of the specific subtopic you studied in papers
\par %Background Provide a context and background information for the readers to understand the motivation and significance of your project. Use statistics, news or other evidence to demonstrate the importance of the project.
\par %Problem Introduce the specific problem that your project aims to solve Explain why this problem is important.
\par %Goal and Contribution: This paragraph offers the main claim of your project and previews what the readers will learn in the remainder of this report. It should also state the scope of your project.
\par% Outline: Give a roadmap of the rest of this report.

\newpage
\section{Project Description}

\subsection{Overview}
\par 

\newpage
\subsection{Method}


\newpage
\section{Demo/Evaluation}
\subsection{Experimental Setup}

\subsection{Results}


\section{Conclusion and Future Work}






\newpage
\section{References}
\printbibliography

\end{document}